건축 공정은 다양한 Stakeholder들이 참여하여 서로 협업을 수행하여 달성된다. 하지만 현재의 건축 현장은 설계자(architect)의 건축 도면이 시공 현장에서 충분히 접근되고 검토되지 않는 문제가 있다. /textit{또한 현장에서의 수정사항이 전체 모델에 일관적으로 반영되기 어렵고, 작업자 사이에서 이러한 변경 정보가 공유되기 어렵다는 문제가 있다.} 이는 현장에서(on-site) 사용하기에는 시스템 구성이 너무 복잡하거나 3차원 모델의 정보가 제공되지 않고, 입력기술이 복잡하여 현장에서의 작업 내역을 반영하기 어렵기 때문이다. 본 논문에서는 이러한 문제점을 해결하기 위하여 포터블 Projection AR 기술을 이용하여 2D와 3D Dual Workspace를 제공하고, IT 기술에 익숙하지 않은 건설-근로자(construction worker)들이 직관적으로 시스템을 사용하도록 하기 위하여 펜과 손가락의 2D/3D Bimanual Interaction를 설계하였다. \textit{또한 모바일 컴퓨팅 기술을 이용하여 작업 중에도 편리하게 전체 건축 모델의 정보를 이용하고 수정사항을 빠르게 모델에 반영하며, 이를 통하여 쉽게 커뮤니케이션이 가능하도록 하였다. 그리고 이러한 시스템을 이용한 Interaction을 Design하였다. 이러한 시스템과 상호작용 설계를 이용하여 건축 관련자들에게 사용하도록 하고 기존의 도면 기반의 방법과 과업(task) 수행 시간을 비교하고 제안하는 시스템의 informal user study를 수행하여 현장에서 유용성이 높음을 확인하였다.}