%!TEX root = ../sigchi-sample.tex


현대 건축 분야는 새로운 컴퓨팅 기술을 이용하여 architect의 창의성을 높이고 다양한 stakeholder 들 간의 커뮤니케이션을 지원하기 위한 다양한 연구들이 이루어지고 있다. 하지만 여전히 건축 현장의 작업자들은 여러 현실적인 문제로 old-fashion 으로 작업을 진행하고 있다\cite{behzadan_visualization_2005}. 이에 따라 설계자(architect)의 의도가 실제 건축물에 충분히 반영되지 못하며, 이러한 과정에서 경제적, 안전적, 설계적 문제점들이 발생할 가능성이 있다. 따라서 이러한 문제점들을 해결하기 위하여 건축 현장에서(on-site) 향상된 IT 기술을 이용하여 실시간으로 건축 정보를 획득하고 이용하기 위한 연구들이 진행되고 있다. 이러한 기술들은 현장(on-site)에서 적용이 가능하도록 포터블하거나 웨어러블 형태로 설계되어야 하며\cite{song_penlight:_2009, yeh_-site_2012}, 현실 세계의 3차원 공간에 쉽게 매핑할 수 있도록 3차원 데이터를 직관적으로 제공하여야 한다\cite{chi_research_2013, kim_interactive_2012, yeh_-site_2012}. 또한, 현장에서 발생되는 수정사항이나 업데이트 정보를 실시간으로 전체 모델에 반영하도록 수정사항을 입력하기 위한 인터페이스도 필요하며\cite{song_penlight:_2009}, 이러한 입력 정보를 통합하여 전체 모델의 정보가 consistent하게 유지하도록 하는 integrated BIM 기술도 중요하다.

본 논문은 이러한 문제점을 극복하고 건축 현장에서 실제로 활용이 가능한 시스템을 제공하기 위하여 3D Interactive Surface 기술\cite{grossman__2010}과 Mobile Computing 기술을 활용하였다. 특히, 3D Interactive Surface 기술의 문제점인 Portablity 문제를 해결하기 위하여 모바일 프로젝터와 깊이 인식 카메라를 사용하였다. 이를 이용하여 작업 환경의 'L'-shape 벽면에 걸쳐 프로젝션함으로써 2D Interaction 을 위한 Horizontal Screen과 3D Interaction을 위한 Vertical Screen을 구성하였다. 이러한 환경에서 NUI 기술을 이용하여 직관적으로 3차원 정보를 조작하고 상호작용 하도록 하였으며, Anoto Pen 기술로 도면 정보를 실시간으로 반영하도록 하였다. 또한, 모바일 폰을 이용하여 작업 중에도 건축 정보에 접근할 수 있도록 하였다. 이러한 시스템을 이용하여, Shared workspace로써의 3D Interactive Surface의 정보와 Personal workspace로써의 모바일 폰의 정보 간의 seamless한 정보의 공유 및 integration을 제공한다. 이를 통하여 건설 현장에서도 seamless하게 건축 정보를 접근하면서 실시간으로 건축 모델의 정보를 업데이트하고, 작업 중에도 편리하게 협업할 수 있는 시스템을 설계하였다. 본 논문에서는 이러한 시스템을 이용하여 건축 현장에 적용하기 위한 design을 수행하고, 이를 건축 관련자들에게 사용 후 informal user study와 비교 연구를 적용하여 시스템의 유용성을 검증하였다. 추가적으로 실험의 피드백과 구현 상의 문제들을 정리하고 결론을 도출하였다.