%!TEX root = ../sigchi-sample.tex


\subsection{Computer Aided Construction}

\begin{comment}
In \cite{harrison_architects_design_????}, the construction process was divided into three stages preliminary design, construction documents, and construction control. As IT technology has developed, various studies have been done to heighten work efficiency in these areas.

First, in the preliminary design phase, there have been studies conducted in helping architects to realize their own conceptual designs into preliminary designs \cite{bae_ilovesketch:_2008, igarashi_teddy:_2007, song_modelcraft_2009, yu_prototype_2007, zeleznik_sketch:_2007}.
These studies use natural interfaces like pens or hands instead of complex interfaces based on existing WIMP (Windows, Icons, Menus, and Pointer). Through them, by executing prototyping faster and more naturally, they allow architects to focus on creative concept design rather than the task of drawing. 

Second, technologies for converting preliminary designs into construction documents have been studied and available in commercial computer-aided design (CAD) software like \textit{AutoCAD, 3dsMax, and SketchUp}.
%\cite{autodesk_3ds_????, autodesk_autocad_????, google_sketchup_????}.

Third, in studies of the construction administration area, using previously made construction documents, studies to aid the course of applying them to construction sites are being done. From studies \cite{behzadan_enabling_2013, kim_interactive_2012}, simulate the positioning of structures and equipment use at sites using augmented reality technology to prevent actual conflicts when structures are built physically.

These studies use 3D objects that are previously manipulated off-site and results from the simulation can detect risk elements realistically but this type of AR tool is not intended for on-site use. Other studies similarly have been used to provide safe and correct construction work based on real-time construction information at work sites. In \cite{chen_framework_2011, giretti_design_2009}, system frameworks were designed in order to access necessary information using mobile phones. These studies can provide necessary information of construction workers by using diverse sensors of smart phones as context aware data. But, these studies have limitations like difficulty in manipulating 3D data due to use of small interfaces of smart phones, restriction to both hands, and difficulty in feedback renewal from the inconvenience of the input interfaces. To overcome the system limits, studies to provide information by a method based on wearable computing were proposed as in \cite{behzadan_visualization_2005, khoury_high-precision_2009, yeh_-site_2012}.

In using wearable computing devices, there are big benefits that additional information can be gained in an environment giving freedom to use of both hands and an increased degree of immersion. However, interfaces to reflect amended matters on-site pose a challenge and previous methods have proved to be inconvenient or not available at all. Song, et al. \cite{song_penlight:_2009, song_mouselight:_2010} solved input interface problems using interactive surface technology. In the study, minimization of projectors was predicted and therefore, through it, by proposing a small projector concept possible to be attached to a pen or with a mouse type, using the projector attached to this pen, information of blueprints were made to be acquired and necessary amended matters were made to be renewed. Nonetheless, because this study is limited to 2D surface, it was difficult to get 3D information of structures or access them.
\end{comment}


The construction process was divided into three stages: preliminary design, construction documents, and construction control. As IT technology has developed, various studies have been conducted to increase work efficiency in these areas.

First, in the preliminary design phase, there have been studies conducted on helping architects to transform their conceptual designs into preliminary designs \cite{bae_ilovesketch:_2008, igarashi_teddy:_2007}. These studies use natural interfaces such as pens or hands, rather than complex interfaces based on windows, icons, menus, and pointer. By enabling faster and more natural prototyping and execution, these interfaces allow architects to focus on creative concept design, rather than the task of drawing.

Second, technologies for converting preliminary designs into construction documents have been studied and are available in commercial computer-aided design (CAD) software such as AutoCAD, 3dsMax, and SketchUp.

Third, in studies of construction administration, studies aimed at aiding in the use of applying previously created construction documents to construction sites are being conducted. Some studies \cite{behzadan_enabling_2013, kim_interactive_2012} simulate the positioning of structures and equipment use at sites using AR technology to prevent actual conflicts when structures are built physically.

These studies use 3D objects previously manipulated off site, and results from the simulation can detect risk elements realistically; however, this type of AR tool is not intended for on-site use. Other studies have been used similarly to provide safe and correct construction work based on real-time construction information at work sites. In \cite{chen_framework_2011, giretti_design_2009}, system frameworks were designed to access required information using mobile devices. These can provide required information regarding construction workers using diverse smart phone sensors as context-aware data. However, these studies have limitations, such as difficulty in manipulating 3D data owing to use of small smart phone interfaces, restriction to both hands, and difficulty in feedback renewal resulting from the inconvenience of the input interfaces. To overcome system limitations, studies of providing information using a method based on wearable computing have been proposed, as in \cite{behzadan_visualization_2005, yeh_-site_2012}.

Using wearable computing devices provides the substantial benefit that additional information can be gained in an environment with free use of both hands and increased immersion. However, interfaces for reflecting amended matters on site pose a challenge, and previous methods have proved to be inconvenient or unavailable. Song et al. \cite{song_penlight:_2009, song_mouselight:_2010} solved input interface problems using interactive surface technology. In that study, minimization of projectors was proposed, and using a small projector that could be attached to a pen or a mouse-type device enabled information regarding blueprints to be acquired and necessary changes to be renewed. Nonetheless, because this study was limited to 2D surfaces, it was difficult to get 3D information regarding the structures or to access them.


%%%%%%%%%%%%%%%%%%%%%%%%%%%%%%%%%%%%%%%%%%%%%%%%%%%%%%%%%%%%%%%%
%%%%%%%%%%%%%%%%%%%%%%%%%%%%%%%%%%%%%%%%%%%%%%%%%%%%%%%%%%%%%%%%
%%%%%%%%%%%%%%%%%%%%%%%%%%%%%%%%%%%%%%%%%%%%%%%%%%%%%%%%%%%%%%%%
%%%%%%%%%%%%%%%%%%%%%%%%%%%%%%%%%%%%%%%%%%%%%%%%%%%%%%%%%%%%%%%%
%%		Interactive 3D Smart Spaces

\subsection{Interactive 3D Smart Spaces}

\begin{comment}
Interactive "smart spaces" technology is a technology to provide computer-based interactivity to general environment normally by using a projector and a camera \cite{kane_bonfire:_2009}. Since a projector additionally provides the function of display to a user environment, it is an integral device in composing this interactive smart space. These kinds of studies started from Wellner's DigitalDesk \cite{wellner_digitaldesk_1991, wellner_interacting_1993}, and diverse studies like Augmented Surface \cite{rekimoto_augmented_1999} and Enhanced Desk \cite{koike_integrating_2001} were developed. Generally, in these studies, researches interacting with 2-dimentional objects like books, paper, walls, and so forth in fixed project environment were mostly realized.

Then, Everywhere Display \cite{pinhanez_everywhere_2001, pinhanez_creating_2003, sukaviriya_portable_2004} designed a steerable projector and proposed that several spaces in rooms could be used dynamically. This dynamic projection smart space technology has been developed into a mobile type in PlayAnywhere system \cite{wilson_playanywhere:_2005}. Later, owing to the minimization of projector devices and as a handheld-sized projector emerged these studies have been possible \cite{cao_interacting_2006, raskar_rfig_2004}. Especially, PenLight \cite{song_penlight:_2009} proposed various interaction methods while providing visions on small projectors to such an extent that it could be possible for them to attach to pens. These studies have later been developed into diverse pen-attached smart space technologies. \cite{kim_ar_2013} proposed a scenario for designing in-situ using this portability and \cite{kim_ar_2014} proposed a system that students can do interactive learning only with pens applying to a learning environment where pens are used mostly. Besides, diverse portable smart space technologies appear like the handheld type \cite{huber_lightbeam:_2012, kim_ilight:_2010} or an embedded type to notebook computers \cite{kane_bonfire:_2009}. The proposed Port3DAr also uses a small portable projector in order to be used at construction sites and was designed to be a standing type on a tripod for a convenient standing.

In these smart space technologies, the limit restricted to 2D plane exists. Recent studies have suggested methods to realize interactive smart spaces in 3D space without this restriction \cite{grossman__2010}. To provide 3D contents on 2D planes, studies [] using a number of additional devices have been developed, but because such systems used in those studies resulted in becoming bulky and complex, it is difficult to apply them to construction sites. In particular, developed systems like \cite{weiss_benddesk:_2010, coram_astrotouch:_2013, wimmer_curve:_2010, benko_miragetable:_2012} could be used on construction sites assuming the availability of dual displays for vertical and horizontal display which can provide 2D and 3D information together. However, because vertical and horizontal displays were composed as independent displays respectively in the existing system, there was a problem of its large bulkiness. Since the proposed Port3DAr uses one projector displayed onto an 'L'-shaped space to interact with shared content, portability can be heightened.

In recent interactive smart space technology, technologies like \cite{jones_illumiroom:_2013, steimle_flexpad:_2013} that can project onto non-planar spaces have been studied well for use in homes and offices but its use is equally applicable to scenarios in complex construction environments. 
\end{comment}


Interactive smart space technology provides computer-based interactivity to a general environment, typically using a projector and a camera \cite{kane_bonfire:_2009}. Since a projector provides the additional function of display to a user environment, it is integral to such an interactive smart space. These types of studies began with Wellner's DigitalDesk \cite{wellner_digitaldesk_1991, wellner_interacting_1993}, followed by diverse studies such as Augmented Surface \cite{rekimoto_augmented_1999} and Enhanced Desk \cite{koike_integrating_2001}. Generally, in these studies, research was conducted regarding interaction with 2D objects such as books, paper, and walls in a fixed project environment.

Subsequently, Everywhere Display \cite{pinhanez_everywhere_2001, sukaviriya_portable_2004} designed a steerable projector and proposed that several spaces in rooms could be used dynamically. This dynamic projection smart space technology has developed into a mobile form in the PlayAnywhere system \cite{wilson_playanywhere:_2005}. Later, owing to the minimization of projection devices and the emergence of handheld projectors, further studies have been possible \cite{cao_interacting_2006, raskar_rfig_2004}. In particular, PenLight \cite{song_penlight:_2009} proposed various interaction methods, while providing vision on small projectors to such an extent that they could be attached to pens. These have since been developed into diverse pen-attached smart space technologies. \cite{kim_ar_2013} proposed a scenario for designing in-situ using this portability, and \cite{kim_ar_2014} proposed a system that enables students to learn interactively using only pens in a learning environment where pens are primarily used. In addition, diverse portable smart space technologies, such as the hand-held type \cite{huber_lightbeam:_2012, kim_ilight:_2010} or the type embedded in notebook computers \cite{kane_bonfire:_2009}, have emerged. The proposed Port3DAr also includes a small portable projector to be used at construction sites and was designed to be a standing type on a tripod for convenience.

These smart space technologies are limited in being restricted to the 2D plane. Recent studies have suggested methods for creating interactive smart spaces in 3D space, without this restriction \cite{grossman__2010}. Studies using various additional devices have been developed to provide 3D content on 2D planes, but because the systems used in those studies are bulky and complex, it is difficult to apply them to construction sites. In particular, developed systems such as those described in \cite{benko_miragetable:_2012, weiss_benddesk:_2010, wimmer_curve:_2010} could be used on construction sites assuming the availability of dual displays for vertical and horizontal display that can provide 2D and 3D information together. However, because vertical and horizontal displays were created as independent displays in the existing systems, bulkiness was a problem. Because the proposed Port3DAr uses one projector displayed onto an L-shaped space to interact with shared content, portability can be increased.

In recent interactive smart space technology, technologies that can project onto non-planar spaces, such as those described in \cite{jones_illumiroom:_2013, steimle_flexpad:_2013}, have been studied extensively for use in homes and offices, but their use is equally applicable to scenarios in complex construction environments.






% end of Related Work
%--------------------------------------------------------------------------------------------
