%-----------------------------------------------------------------------------
%	7. Conclusion and Future Work
%	목표 분량: 0.5장

\section{Conclusion and Future Work}

%본 논문에서는 건축 시공 현장에서 설계 도면이나 3차원 모델과 같은 건축 정보를 실시간으로 접근하고 직관적인 인터페이스를 이용하여 이를 조작할 수 있는 포터블 2D/3D 건축 정보 인터페이스를 제안하였다. 이를 통하여 시공 현장에서 전체 모델에 대한 이해를 높일 수 있고, 특정 영역에 대한 정보를 쉽게 획득할 수 있으며, 현장에서의 수정 사항을 즉각적으로 모델에 재반영함으로써 협업의 효율을 높일 수 있었다. 특히 이를 포터블 형태의 기기에서 구현함으로써 실제 현장에서 사용이 가능하다는 장점이 있었다.

% This paper approached construction information like blueprints or 3D models in real time and proposed a portable 2D/3D construction information interface using intuitive projection design and interaction to manipulate content. Through this, understanding of all models at construction sites can be enhanced and information about special regions can be acquired easily. In addition, by re-reflecting modified matters on models promptly, efficiency of cooperation could be heightened. Particularly, by implementing them as portable, participants strongly agreed that our system could be beneficial on real construction sites.
%하지만, NUI 인식 기술 등을 향상하여 전체 시스템의 구성을 단순화하는 필요성이 있으며, HMD 등의 웨어러블 기기를 이용한 상호작용 지원 등을 통하여 실시간성과 사용성을 높일 수 있는 다른 플랫폼 과의 협업도 중요할 것으로 생각된다. 또한 Revit과 같은 상용 BIM Software와의 연동을 제공함으로써 좀 더 상세한 건축 정보 제공한다면 더욱 편리한 인터페이스로 발전할 수 있을 것으로 기대된다.
% However, there is a need to simplify overall system construction through improvements in NUI recognition technology, and cooperation with other platforms to increase usability and speed through support for interactions using wearable devices such as HMDs is important.  Development of more convenient interfaces can be expected if more detailed construction information is offered through interlocking with commercial BIM software such as Revit. 

\begin{comment}
하지만 추가적으로 HMD 기반의 웨어러블 증강현실이나 모바일 증강현실 시스템과의 연동 기능을 제공하여 현재 작업 Context에 적합한 정보를 제공하는 연구를 진행하는 것이 필요할 것이다. 또한 Revit과 같은 상용 BIM Software와의 연동을 제공함으로써 좀 더 상세한 건축 정보 제공한다면 더욱 편리한 인터페이스로 발전할 수 있을 것으로 기대된다.

%But, for this system to work providing the interlocking functions with wearable or mobile augmented reality systems is necesary and further study is needed to process contexts in the present work. Plus, by providing interlock with common BIM software like revit, if more detailed construction information is provided, Port3DAr can be developed into a more convenient interface. 
\end{comment}

This paper addressed construction information such as blueprints or 3D models in real time and proposed a portable 2D/3D construction information interface using intuitive projection design and interaction to manipulate content. This can enhance understanding of models at construction sites and enables easy acquisition of information regarding specific regions. In addition, cooperation efficiency can be increased by prompt re-reflection of model modifications. In particular, participants strongly agreed that our system could be beneficial on real construction sites as a result of portable implementation. However, there is a need to simplify overall system construction through improvements in NUI recognition technology, and cooperation with other platforms to increase usability and speed through support for interaction using wearable devices such as head-mounted displays is important. Development of more convenient interfaces can be expected, if more detailed construction information is offered through interlocking with commercial BIM software such as Revit.